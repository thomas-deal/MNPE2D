\section{Introduction}

In the early 1990s, a numerical code known as the University of Miami Parabolic Equation (UMPE) Model was documented and made available to the general research community\cite{Smith1993}. This model was based on the split-step Fourier (SSF) technique\cite{Hardin1973}, and had been adapted from previous versions developed by Fred Tappert at the University of Miami. A subsequent version, known as the Monterey-Miami Parabolic Equation (MMPE) Model, was developed in the mid 1990s that was more streamlined and user-friendly. This code was thoroughly tested against several existing benchmark scenarios and was found to perform reasonably well during the Shallow Water Acoustic Modeling Workshop help in Monterey, CA in 1999 (SWAM'99)\cite{Smith2001}. Subsequent upgrades to the model included a correction to the treatment of bottom loss\cite{Smith2007}, implementation of the complex density approach to handle shear wave losses\cite{Zhang1995}, and the computation of horizontal and vertical velocity fields\cite{Smith2008}. The latest version of the model, known as the Monterey-Newport Parabolic Equation (MNPE) Model, includes built-in support for a variety of rough ocean surfaces\cite{Tappert} and built-in horizontal and vertical dipole sources, which enable reciprocal vector field calculations\cite{Deal2017}.

This document provides an introduction to version 1.0 of the broadband, two-dimensional MNPE model (MNPE2D). It describes the steps required to build the executable from source code, generate the input files, run the model, and read the outputs. It also includes example outputs the user can re-create to verify the model is running correctly on their system.
