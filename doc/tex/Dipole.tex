\section{Dipole Sources in MNPE}\label{app:dipolesource}

Generating fields for dipole sources is accomplished by modifying the PE starter field. Vecherin, et al.\cite{Vecherin2011} prescribes an equivalent source method that decomposes vertical and horizontal dipole sources into vertical monopole arrays with element amplitude scaling such that the array source strength is equal to that of a single monopole.

A vertical dipole is a special case of a vertical array with two elements 180 degrees out of phase with each other and separated by a distance $2d_z$ much smaller than a wavelength ($kd_z\leq0.1$). The desired shape of the acoustic field for this source is $\cos(\phi)$ for angles $\phi$ measured from the vertical axis. The horizontal dipole has a desired shape of $\sin(\phi)$, which Vecherin represents as a sum of cosines for angles close to $\pi/2$,
\begin{equation}
\sin(\phi) \approx 1 - \frac{1}{2}\cos^2(\phi) - \frac{1}{8}\cos^4(\phi),\label{eq:sineapprox}
\end{equation}
which is valid for wide-angle PE calculations. The $\cos^n(\phi)$ terms in \eqnname~(\ref{eq:sineapprox}) can be represented by arrays of 1, 3, and 5 elements vertically spaced by $2d_z$, which when added together produce the desired PE starter field. Although these amplitude weights and element spacings can be implemented with MNPE's existing vertical array source, the utility of dedicated horizontal and vertical dipole sources led to a direct implementation of the starter field in the vertical wavenumber domain such that the user need only specify the source depth and frequency.

Each of these point sources is implemented with an analytic starter, which accounts for reflection from a pressure release surface. The analytic starter does not account for reflection from the sea floor, which can be important in shallow water or when the source is close to the bottom. The dipole starters are further limited by the vertical extent of their equivalent arrays: when the source is located near the surface or bottom, some array elements may be on the other side of the boundary, and they do not produce the correct acoustic fields. These errors are more pronounced for sources near the surface than for sources near the bottom, since the depth mixing function at the bottom boundary introduces a smoother transition than the perfectly reflecting ocean surface. A modal starter may be more appropriate for dipole sources near surface and bottom boundaries in some cases.
